\documentclass{article}
\usepackage{geometry}
\usepackage[outputdir=cache]{minted} 
% \geometry{margin=1in}


\usepackage{amsmath,amssymb,amsfonts,amsthm} % Related to math
\usepackage{mdframed, enumitem}
\usepackage[utf8]{inputenc} 
\inputencoding{latin1}
\inputencoding{utf8} 
\setlist[enumerate]{itemsep=0mm} % no spacing bw enumerations
\usepackage[parfill]{parskip} % new line paragraph style
\usepackage{tabularx}
\usepackage{cite} 
\usepackage{tabularx}
\usepackage{graphicx} 
\usepackage{hyperref}



\title{Research}
\author{Andrew Gazelka\\ \url{gazel007@umn.edu}}
\date{\today}

\newcommand{\formatheadingtext}[1]{
    \textbf{Problem\ #1}:
}

\newenvironment{newproblem}[1]
    {\begin{mdframed} \formatheadingtext{#1}\ignorespaces}{\end{mdframed} \vspace{5mm}} 
\begin{document}

\maketitle

\begin{itemize}
    \item \href{https://en.wikipedia.org/wiki/Quadtree#:~:text=A%20quadtree%20is%20a%20tree,into%20four%20quadrants%20or%20regions.}{Quadtrees, Wiki}
\end{itemize}

Parsing was rather difficult because of storage the protobuf \verb|map.osm| file was only 195M, 
however a naïve method of putting it into memory made it store over 32GB. 

Originally, we had a map of node locations \verb|id -> location| and a map of node to its connections. \verb|id -> id[]|

An id is i64. Ideally this could be a lot smaller. 
\begin{center} 
    \begin{tabular}{cc} 
    u16 & 65535
    \end{tabular}
\end{center}
test
\bibliographystyle{unsrt}
\bibliography{main}
\end{document}
